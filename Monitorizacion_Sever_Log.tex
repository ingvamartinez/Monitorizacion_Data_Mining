% Options for packages loaded elsewhere
\PassOptionsToPackage{unicode}{hyperref}
\PassOptionsToPackage{hyphens}{url}
%
\documentclass[
]{article}
\usepackage{amsmath,amssymb}
\usepackage{lmodern}
\usepackage{iftex}
\ifPDFTeX
  \usepackage[T1]{fontenc}
  \usepackage[utf8]{inputenc}
  \usepackage{textcomp} % provide euro and other symbols
\else % if luatex or xetex
  \usepackage{unicode-math}
  \defaultfontfeatures{Scale=MatchLowercase}
  \defaultfontfeatures[\rmfamily]{Ligatures=TeX,Scale=1}
\fi
% Use upquote if available, for straight quotes in verbatim environments
\IfFileExists{upquote.sty}{\usepackage{upquote}}{}
\IfFileExists{microtype.sty}{% use microtype if available
  \usepackage[]{microtype}
  \UseMicrotypeSet[protrusion]{basicmath} % disable protrusion for tt fonts
}{}
\makeatletter
\@ifundefined{KOMAClassName}{% if non-KOMA class
  \IfFileExists{parskip.sty}{%
    \usepackage{parskip}
  }{% else
    \setlength{\parindent}{0pt}
    \setlength{\parskip}{6pt plus 2pt minus 1pt}}
}{% if KOMA class
  \KOMAoptions{parskip=half}}
\makeatother
\usepackage{xcolor}
\usepackage[margin=1in]{geometry}
\usepackage{color}
\usepackage{fancyvrb}
\newcommand{\VerbBar}{|}
\newcommand{\VERB}{\Verb[commandchars=\\\{\}]}
\DefineVerbatimEnvironment{Highlighting}{Verbatim}{commandchars=\\\{\}}
% Add ',fontsize=\small' for more characters per line
\usepackage{framed}
\definecolor{shadecolor}{RGB}{248,248,248}
\newenvironment{Shaded}{\begin{snugshade}}{\end{snugshade}}
\newcommand{\AlertTok}[1]{\textcolor[rgb]{0.94,0.16,0.16}{#1}}
\newcommand{\AnnotationTok}[1]{\textcolor[rgb]{0.56,0.35,0.01}{\textbf{\textit{#1}}}}
\newcommand{\AttributeTok}[1]{\textcolor[rgb]{0.77,0.63,0.00}{#1}}
\newcommand{\BaseNTok}[1]{\textcolor[rgb]{0.00,0.00,0.81}{#1}}
\newcommand{\BuiltInTok}[1]{#1}
\newcommand{\CharTok}[1]{\textcolor[rgb]{0.31,0.60,0.02}{#1}}
\newcommand{\CommentTok}[1]{\textcolor[rgb]{0.56,0.35,0.01}{\textit{#1}}}
\newcommand{\CommentVarTok}[1]{\textcolor[rgb]{0.56,0.35,0.01}{\textbf{\textit{#1}}}}
\newcommand{\ConstantTok}[1]{\textcolor[rgb]{0.00,0.00,0.00}{#1}}
\newcommand{\ControlFlowTok}[1]{\textcolor[rgb]{0.13,0.29,0.53}{\textbf{#1}}}
\newcommand{\DataTypeTok}[1]{\textcolor[rgb]{0.13,0.29,0.53}{#1}}
\newcommand{\DecValTok}[1]{\textcolor[rgb]{0.00,0.00,0.81}{#1}}
\newcommand{\DocumentationTok}[1]{\textcolor[rgb]{0.56,0.35,0.01}{\textbf{\textit{#1}}}}
\newcommand{\ErrorTok}[1]{\textcolor[rgb]{0.64,0.00,0.00}{\textbf{#1}}}
\newcommand{\ExtensionTok}[1]{#1}
\newcommand{\FloatTok}[1]{\textcolor[rgb]{0.00,0.00,0.81}{#1}}
\newcommand{\FunctionTok}[1]{\textcolor[rgb]{0.00,0.00,0.00}{#1}}
\newcommand{\ImportTok}[1]{#1}
\newcommand{\InformationTok}[1]{\textcolor[rgb]{0.56,0.35,0.01}{\textbf{\textit{#1}}}}
\newcommand{\KeywordTok}[1]{\textcolor[rgb]{0.13,0.29,0.53}{\textbf{#1}}}
\newcommand{\NormalTok}[1]{#1}
\newcommand{\OperatorTok}[1]{\textcolor[rgb]{0.81,0.36,0.00}{\textbf{#1}}}
\newcommand{\OtherTok}[1]{\textcolor[rgb]{0.56,0.35,0.01}{#1}}
\newcommand{\PreprocessorTok}[1]{\textcolor[rgb]{0.56,0.35,0.01}{\textit{#1}}}
\newcommand{\RegionMarkerTok}[1]{#1}
\newcommand{\SpecialCharTok}[1]{\textcolor[rgb]{0.00,0.00,0.00}{#1}}
\newcommand{\SpecialStringTok}[1]{\textcolor[rgb]{0.31,0.60,0.02}{#1}}
\newcommand{\StringTok}[1]{\textcolor[rgb]{0.31,0.60,0.02}{#1}}
\newcommand{\VariableTok}[1]{\textcolor[rgb]{0.00,0.00,0.00}{#1}}
\newcommand{\VerbatimStringTok}[1]{\textcolor[rgb]{0.31,0.60,0.02}{#1}}
\newcommand{\WarningTok}[1]{\textcolor[rgb]{0.56,0.35,0.01}{\textbf{\textit{#1}}}}
\usepackage{graphicx}
\makeatletter
\def\maxwidth{\ifdim\Gin@nat@width>\linewidth\linewidth\else\Gin@nat@width\fi}
\def\maxheight{\ifdim\Gin@nat@height>\textheight\textheight\else\Gin@nat@height\fi}
\makeatother
% Scale images if necessary, so that they will not overflow the page
% margins by default, and it is still possible to overwrite the defaults
% using explicit options in \includegraphics[width, height, ...]{}
\setkeys{Gin}{width=\maxwidth,height=\maxheight,keepaspectratio}
% Set default figure placement to htbp
\makeatletter
\def\fps@figure{htbp}
\makeatother
\setlength{\emergencystretch}{3em} % prevent overfull lines
\providecommand{\tightlist}{%
  \setlength{\itemsep}{0pt}\setlength{\parskip}{0pt}}
\setcounter{secnumdepth}{-\maxdimen} % remove section numbering
\ifLuaTeX
  \usepackage{selnolig}  % disable illegal ligatures
\fi
\IfFileExists{bookmark.sty}{\usepackage{bookmark}}{\usepackage{hyperref}}
\IfFileExists{xurl.sty}{\usepackage{xurl}}{} % add URL line breaks if available
\urlstyle{same} % disable monospaced font for URLs
\hypersetup{
  pdftitle={Monitoring\_Server\_Log},
  pdfauthor={Victor Adrian Martinez},
  hidelinks,
  pdfcreator={LaTeX via pandoc}}

\title{Monitoring\_Server\_Log}
\author{Victor Adrian Martinez}
\date{2023-04-25}

\begin{document}
\maketitle

\begin{description}
\item[\href{https://github.com/ingvamartinez/Monitorizacion_Data_Mining}{link
Github}]
\url{https://github.com/ingvamartinez/Monitorizacion_Data_Mining}
\end{description}

\#Actividad: Monitoreo de log de servidor

\hypertarget{actividad-1}{%
\subsection{Actividad 1}\label{actividad-1}}

\hypertarget{anuxe1lisis-de-logs-de-servidor-usando-r}{%
\paragraph{Análisis de logs de servidor usando
R}\label{anuxe1lisis-de-logs-de-servidor-usando-r}}

\hypertarget{pregunta-1}{%
\paragraph{Pregunta 1:}\label{pregunta-1}}

Queremos programar un script con el que podamos hacer una investigación
forense sobre un fichero de logs de un servidor de tipo Apache. Los
datos del registro del servidor están en el formato estándar e incluyen
miles de registros sobre las distintas peticiones gestionadas por el
servidor web.

Nuestro programa ha de ser capaz de obtener las respuestas de forma
dinámica a las siguientes preguntas utilizando instrucciones de código
en R: 1. Descomprimir el fichero comprimido que contiene los registros
del servidor, y a partir de los datos extraídos, cargar en data frames
los registros con las peticiones servidas.

El primer paso para realizartareas de análisis consiste en la carga de
los datos, la exploración inicial y limpieza de los datos (filtrado,
corrección de tipo de datos, ajuste de la forma de los datos, etc.).
Para esto usaremos la capacidad de R y de sus librerías (tidyr y dplyr)
para modificar los data frames cargados y adaptarlos a la estructura de
datos elegantes: cada fila se corresponderá a una observación y cada
columna únicamente una variable.

De esta forma, se facilita el análisis y la posterior modificación de la
forma del data frame, adaptando este para que pueda ser usado en la
mayoría de funciones de R.

\texttt{code}

\begin{Shaded}
\begin{Highlighting}[]
\NormalTok{\{}
\FunctionTok{library}\NormalTok{(dplyr)}
\FunctionTok{library}\NormalTok{(tidyr) }
\FunctionTok{library}\NormalTok{(stringr)}

\CommentTok{\# Obtenemos la Data}

\NormalTok{data\_path}\OtherTok{\textless{}{-}}\StringTok{"epa{-}http.csv"}

\NormalTok{data}\OtherTok{\textless{}{-}}\FunctionTok{read.csv}\NormalTok{(data\_path, }\AttributeTok{col.names =} \FunctionTok{c}\NormalTok{(}\StringTok{\textquotesingle{}Request.IP\textquotesingle{}}\NormalTok{,}\StringTok{\textquotesingle{}Request\_DATE\textquotesingle{}}\NormalTok{,}
\StringTok{\textquotesingle{}Request\_Solicitud\textquotesingle{}}\NormalTok{, }\StringTok{\textquotesingle{}Status\textquotesingle{}}\NormalTok{, }\StringTok{\textquotesingle{}Size\textquotesingle{}}\NormalTok{) , }\AttributeTok{sep=}\StringTok{" "}\NormalTok{, }\AttributeTok{header =} \ConstantTok{FALSE}\NormalTok{)}

\CommentTok{\# Relizamos la siguientes operaciones para obtener un DF con todos los datos}

\NormalTok{data\_sol}\OtherTok{\textless{}{-}}\FunctionTok{as.data.frame}\NormalTok{(data}\SpecialCharTok{$}\NormalTok{V3) }
\NormalTok{data\_sol }\OtherTok{\textless{}{-}} \FunctionTok{str\_split\_fixed}\NormalTok{(data}\SpecialCharTok{$}\NormalTok{Request\_Solicitud,}\StringTok{" "}\NormalTok{,}\DecValTok{3}\NormalTok{) }
\NormalTok{data\_sol }\OtherTok{\textless{}{-}} \FunctionTok{as.data.frame}\NormalTok{(data\_sol) }
\NormalTok{data\_sol }\OtherTok{\textless{}{-}} \FunctionTok{rename}\NormalTok{(data\_sol,}\FunctionTok{c}\NormalTok{(}\AttributeTok{RequestType =}\NormalTok{ V1, }\AttributeTok{RequestUrl =}\NormalTok{ V2, }\AttributeTok{RequestProtocol =}
\NormalTok{V3)) }
\NormalTok{datafull }\OtherTok{\textless{}{-}} \FunctionTok{cbind.data.frame}\NormalTok{(data,data\_sol)}

\FunctionTok{head}\NormalTok{(datafull,}\DecValTok{2}\NormalTok{) \}}
\end{Highlighting}
\end{Shaded}

\begin{verbatim}
## 
## Attaching package: 'dplyr'
\end{verbatim}

\begin{verbatim}
## The following objects are masked from 'package:stats':
## 
##     filter, lag
\end{verbatim}

\begin{verbatim}
## The following objects are masked from 'package:base':
## 
##     intersect, setdiff, setequal, union
\end{verbatim}

\begin{verbatim}
##                Request.IP  Request_DATE           Request_Solicitud Status Size
## 1           141.243.1.172 [29:23:53:25] GET /Software.html HTTP/1.0    200 1497
## 2 query2.lycos.cs.cmu.edu [29:23:53:36] GET /Consumer.html HTTP/1.0    200 1325
##   RequestType     RequestUrl RequestProtocol
## 1         GET /Software.html        HTTP/1.0
## 2         GET /Consumer.html        HTTP/1.0
\end{verbatim}

\hypertarget{pregunta-2}{%
\paragraph{Pregunta 2:}\label{pregunta-2}}

Explorar el contenido del fichero descomprimido y cargado en un data
frame. Identificar el número único de usuarios que han interactuado
directamente con el servidor de forma segregada según si los usuarios
han tenido algún tipo de error en las distintas peticiones ofrecidas por
el servidor.Idealmente, hacer el break down del número de usuarios en
función de si estos han tenido algún tipo de error durante las
interacciones con el servidor, es decir, ofrecer el número de usuarios
que no han tenido ningún error en una de las peticiones gestionadas por
el servidor y el caso contrario, el número de usuarios que sí han
experimentado algún error para una petición servida.

Para determinar si una respuesta ha sido servida de forma satisfactoria
o no se puede usar el codigo de retorno de la petición. Los códigos de
retorno pertenecientes a familia de los 200, se pueden considerar
peticiones servidas correctamente.

\texttt{code}

\begin{Shaded}
\begin{Highlighting}[]
\FunctionTok{print}\NormalTok{(}\StringTok{\textquotesingle{}Numero de Ips sin errores\textquotesingle{}}\NormalTok{)}
\end{Highlighting}
\end{Shaded}

\begin{verbatim}
## [1] "Numero de Ips sin errores"
\end{verbatim}

\begin{Shaded}
\begin{Highlighting}[]
\CommentTok{\#Seleccionamos las variables que necesitamos}
\NormalTok{ip\_errors}\OtherTok{\textless{}{-}} \FunctionTok{subset}\NormalTok{(datafull, }\AttributeTok{select=}\FunctionTok{c}\NormalTok{(Request.IP,Status))}
\NormalTok{ip\_errors}\OtherTok{\textless{}{-}} \FunctionTok{as.data.frame}\NormalTok{(ip\_errors)}
\CommentTok{\#Agregamos una columna para clasificar si tiene error}
\NormalTok{ip\_errors}\SpecialCharTok{$}\NormalTok{error }\OtherTok{\textless{}{-}} \FunctionTok{ifelse}\NormalTok{(ip\_errors}\SpecialCharTok{$}\NormalTok{Status }\SpecialCharTok{!=} \DecValTok{200}\NormalTok{,}\DecValTok{1}\NormalTok{,}\DecValTok{0}\NormalTok{)}
\NormalTok{ip\_errors}\OtherTok{\textless{}{-}}\FunctionTok{as.data.frame}\NormalTok{(}\FunctionTok{subset}\NormalTok{(ip\_errors,error }\SpecialCharTok{==} \DecValTok{1}\NormalTok{))}
\NormalTok{g\_errors}\OtherTok{\textless{}{-}}\NormalTok{ ip\_errors }\SpecialCharTok{\%\textgreater{}\%}  \FunctionTok{group\_by}\NormalTok{(Request.IP) }\SpecialCharTok{\%\textgreater{}\%} 
   \FunctionTok{summarise}\NormalTok{(}\AttributeTok{sum =} \FunctionTok{sum}\NormalTok{(error))}
\CommentTok{\#Realizamos la suma aritmetica para obtener la cantidad de los usuarios con errores y sin errores}
\NormalTok{IP\_error}\OtherTok{\textless{}{-}} \FunctionTok{length}\NormalTok{(}\FunctionTok{unique}\NormalTok{(g\_errors}\SpecialCharTok{$}\NormalTok{Request.IP))}
\NormalTok{IP\_total}\OtherTok{\textless{}{-}} \FunctionTok{length}\NormalTok{(}\FunctionTok{unique}\NormalTok{(datafull}\SpecialCharTok{$}\NormalTok{Request.IP))}
\NormalTok{IP\_noerror }\OtherTok{\textless{}{-}}\NormalTok{ IP\_total}\SpecialCharTok{{-}}\NormalTok{IP\_error}
\FunctionTok{print}\NormalTok{(}\FunctionTok{paste}\NormalTok{(}\StringTok{\textquotesingle{}Usuarios Totales=\textquotesingle{}}\NormalTok{,IP\_total))}
\end{Highlighting}
\end{Shaded}

\begin{verbatim}
## [1] "Usuarios Totales= 2333"
\end{verbatim}

\begin{Shaded}
\begin{Highlighting}[]
\FunctionTok{print}\NormalTok{(}\FunctionTok{paste}\NormalTok{(}\StringTok{\textquotesingle{}Usuarios sin errores=\textquotesingle{}}\NormalTok{, IP\_noerror))}
\end{Highlighting}
\end{Shaded}

\begin{verbatim}
## [1] "Usuarios sin errores= 1089"
\end{verbatim}

\begin{Shaded}
\begin{Highlighting}[]
\FunctionTok{print}\NormalTok{(}\FunctionTok{paste}\NormalTok{(}\StringTok{\textquotesingle{}Usuarios con errores=\textquotesingle{}}\NormalTok{, IP\_error))}
\end{Highlighting}
\end{Shaded}

\begin{verbatim}
## [1] "Usuarios con errores= 1244"
\end{verbatim}

\hypertarget{pregunta-3}{%
\paragraph{Pregunta 3:}\label{pregunta-3}}

Analizar los distintos tipos de peticiones HTTP (GET, POST, PUT, DELETE)
gestionadas por el servidor, identificando la frecuencia de cada una de
estas. Repetir el análisis, esta vez filtrando previamente aquellas
peticiones correspondientes a recursos ofrecidos de tipo imagen.

\texttt{code}

\begin{Shaded}
\begin{Highlighting}[]
\CommentTok{\#Pregunta 3}

\CommentTok{\#Analizar los distintos tipos de peticiones HTTP (GET, POST, PUT, DELETE) }
\CommentTok{\#gestionadas por el servidor, identificando la frecuencia de cada una de estas.}

\NormalTok{frec\_Prot\_http}\OtherTok{\textless{}{-}} \FunctionTok{data.frame}\NormalTok{(}\FunctionTok{table}\NormalTok{(datafull}\SpecialCharTok{$}\NormalTok{RequestType))}
\FunctionTok{head}\NormalTok{(frec\_Prot\_http)}
\end{Highlighting}
\end{Shaded}

\begin{verbatim}
##   Var1  Freq
## 1  GET 46020
## 2 HEAD   106
## 3 POST  1622
\end{verbatim}

\begin{Shaded}
\begin{Highlighting}[]
\CommentTok{\#Repetir el análisis, esta vez filtrando previamente aquellas peticiones }
\CommentTok{\#correspondientes a recursos ofrecidos de tipo imagen.}
\NormalTok{tipo\_imagen}\OtherTok{\textless{}{-}} \FunctionTok{grepl}\NormalTok{(}\AttributeTok{pattern =} \StringTok{".*[png|jpg|gif|ico]$"}\NormalTok{, datafull}\SpecialCharTok{$}\NormalTok{RequestUrl)}
\NormalTok{tipo\_imagen}\OtherTok{\textless{}{-}} \FunctionTok{as.data.frame}\NormalTok{(tipo\_imagen)}
\NormalTok{datafull2}\OtherTok{\textless{}{-}}\FunctionTok{cbind.data.frame}\NormalTok{(datafull,tipo\_imagen)}
\NormalTok{table\_df2}\OtherTok{\textless{}{-}}\NormalTok{ datafull2 }\SpecialCharTok{\%\textgreater{}\%} \FunctionTok{select}\NormalTok{(RequestType,tipo\_imagen)}
\NormalTok{table\_df2}\OtherTok{\textless{}{-}} \FunctionTok{as.data.frame}\NormalTok{(}\FunctionTok{subset}\NormalTok{(table\_df2,tipo\_imagen }\SpecialCharTok{==} \ConstantTok{TRUE}\NormalTok{))}
\NormalTok{frec\_prot\_img}\OtherTok{\textless{}{-}} \FunctionTok{data.frame}\NormalTok{(}\FunctionTok{table}\NormalTok{(table\_df2}\SpecialCharTok{$}\NormalTok{RequestType))}
\FunctionTok{head}\NormalTok{(frec\_prot\_img)}
\end{Highlighting}
\end{Shaded}

\begin{verbatim}
##   Var1  Freq
## 1  GET 22676
## 2 HEAD    57
## 3 POST   232
\end{verbatim}

\hypertarget{pregunta-4}{%
\paragraph{Pregunta 4:}\label{pregunta-4}}

Generar un gráfico que permita visualizar las respuestas del servidor,
es decir, la distribución de peticiones según el código de respuesta de
esta.

\texttt{code}

\begin{Shaded}
\begin{Highlighting}[]
\CommentTok{\# Graficos}

\FunctionTok{library}\NormalTok{(ggplot2)}
\NormalTok{df4 }\OtherTok{\textless{}{-}} \FunctionTok{as.data.frame}\NormalTok{(datafull)}
\NormalTok{df4}\SpecialCharTok{$}\NormalTok{Status }\OtherTok{\textless{}{-}} \FunctionTok{as.factor}\NormalTok{(df4}\SpecialCharTok{$}\NormalTok{Status)}
\CommentTok{\#1}
\NormalTok{gf\_Hist}\OtherTok{\textless{}{-}}\FunctionTok{ggplot}\NormalTok{(df4, }\FunctionTok{aes}\NormalTok{(Status, }\AttributeTok{fill =}\NormalTok{ Status))  }\SpecialCharTok{+} \FunctionTok{geom\_histogram}\NormalTok{(}\AttributeTok{bin=}\DecValTok{10}\NormalTok{,}\AttributeTok{stat =} \StringTok{"count"}\NormalTok{)}\SpecialCharTok{+}
  \FunctionTok{labs}\NormalTok{(}\AttributeTok{title=}\StringTok{"Grafico Histograma"}\NormalTok{)}
\end{Highlighting}
\end{Shaded}

\begin{verbatim}
## Warning in geom_histogram(bin = 10, stat = "count"): Ignoring unknown
## parameters: `binwidth`, `bins`, `pad`, and `bin`
\end{verbatim}

\begin{Shaded}
\begin{Highlighting}[]
\NormalTok{gf\_Hist}
\end{Highlighting}
\end{Shaded}

\includegraphics{Monitorizacion_Sever_Log_files/figure-latex/unnamed-chunk-5-1.pdf}

\begin{Shaded}
\begin{Highlighting}[]
\CommentTok{\#2}
\NormalTok{gf\_Bar}\OtherTok{\textless{}{-}}\FunctionTok{ggplot}\NormalTok{(df4, }\FunctionTok{aes}\NormalTok{(Status, }\AttributeTok{fill =}\NormalTok{ Status)) }\SpecialCharTok{+} \FunctionTok{geom\_bar}\NormalTok{(}\AttributeTok{stat =} \StringTok{"count"}\NormalTok{)}\SpecialCharTok{+}
  \FunctionTok{labs}\NormalTok{(}\AttributeTok{title=}\StringTok{"Gráfico de Barras"}\NormalTok{)}
\NormalTok{gf\_Bar}
\end{Highlighting}
\end{Shaded}

\includegraphics{Monitorizacion_Sever_Log_files/figure-latex/unnamed-chunk-6-1.pdf}

\begin{Shaded}
\begin{Highlighting}[]
\CommentTok{\#3}
\NormalTok{sumVec}\OtherTok{\textless{}{-}}\FunctionTok{as.character}\NormalTok{(df4}\SpecialCharTok{$}\NormalTok{Status)}
\NormalTok{table\_status}\OtherTok{\textless{}{-}}\FunctionTok{table}\NormalTok{(sumVec)}
\NormalTok{df5}\OtherTok{\textless{}{-}}\FunctionTok{data.frame}\NormalTok{(table\_status)}
\NormalTok{df5}\OtherTok{\textless{}{-}}\FunctionTok{rename}\NormalTok{(df5,}\FunctionTok{c}\NormalTok{(}\AttributeTok{Status=}\NormalTok{sumVec, }\AttributeTok{Url=}\NormalTok{Freq))}
\NormalTok{etiquetas }\OtherTok{\textless{}{-}} \FunctionTok{paste0}\NormalTok{(df5}\SpecialCharTok{$}\NormalTok{Status,}\StringTok{"="}\NormalTok{,}\FunctionTok{round}\NormalTok{(}\DecValTok{100} \SpecialCharTok{*}\NormalTok{ df5}\SpecialCharTok{$}\NormalTok{Url}\SpecialCharTok{/}\FunctionTok{sum}\NormalTok{(df5}\SpecialCharTok{$}\NormalTok{Url), }\DecValTok{2}\NormalTok{), }\StringTok{"\%"}\NormalTok{)}

\NormalTok{gf\_Pie}\OtherTok{\textless{}{-}} \FunctionTok{ggplot}\NormalTok{(df5,}\FunctionTok{aes}\NormalTok{(}\AttributeTok{x=}\StringTok{" "}\NormalTok{,}\AttributeTok{y=}\NormalTok{Url, }\AttributeTok{fill=}\NormalTok{ Status))}\SpecialCharTok{+}
  \FunctionTok{geom\_bar}\NormalTok{(}\AttributeTok{stat =} \StringTok{"identity"}\NormalTok{,}\AttributeTok{color=}\StringTok{"white"}\NormalTok{ )}\SpecialCharTok{+}
  \FunctionTok{geom\_text}\NormalTok{(}\FunctionTok{aes}\NormalTok{(}\AttributeTok{label=}\NormalTok{etiquetas),}
            \AttributeTok{position=}\FunctionTok{position\_stack}\NormalTok{(}\AttributeTok{vjust=}\FloatTok{0.5}\NormalTok{),}\AttributeTok{color=}\StringTok{"white"}\NormalTok{,}\AttributeTok{size=}\DecValTok{2}\NormalTok{)}\SpecialCharTok{+}
  \FunctionTok{coord\_polar}\NormalTok{(}\AttributeTok{theta=}\StringTok{"y"}\NormalTok{)}\SpecialCharTok{+}
  \FunctionTok{theme\_void}\NormalTok{()}\SpecialCharTok{+}
  \FunctionTok{labs}\NormalTok{(}\AttributeTok{title=}\StringTok{"Gráfico de Pie"}\NormalTok{)}
\NormalTok{gf\_Pie}
\end{Highlighting}
\end{Shaded}

\includegraphics{Monitorizacion_Sever_Log_files/figure-latex/unnamed-chunk-7-1.pdf}

```

\end{document}
